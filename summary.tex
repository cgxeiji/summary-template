\documentclass{kdl-summary}

\usepackage[T1]{fontenc}
\usepackage{textcomp}
\usepackage{graphicx}
\usepackage{balance}
\usepackage{booktabs}
\usepackage{ragged2e}
\usepackage{multicol}
\usepackage[
    backend=biber,
    style=authoryear
]{biblatex}
\addbibresource{reference.bib}


% Title of the article
\title{KDL Paper Summary Sample File}
% Your name
\author{Firstname Lastname}

\begin{document}
\nocite{*}

\twocolumn[\maketitle]

% Include an icon that will help us remember the study (you can make it
% yourself or search on the internet). The idea is to make a visual connection
% with the research so we can say "Remember the paper with the cute cat? We can
% use that analysis!".
\memoryicon{icon.png}


\begin{abstract}
    We are going to use this template to summarize papers. This summary should
    use easy language, no jargon allowed! (unless you explain it).
    It has 4 sections, the \textbf{Abstract}, where you will copy/paste the
    abstract of the paper; the \textbf{What's So Special} which says what is
    new in the paper, what is the hypothesis, and the final conclusions; the
    \textbf{Interesting Tools and Methods} section where you can mention surveys,
    analysis, techniques, etc., used in the study; and the \textbf{You Should
        Also Know!} section where you can put things that caught your eye, like
    theories, definitions, anecdotes, among others.
\end{abstract}

\begin{novelty}
    Make a summary with your own words.
    Also, include the \textbf{hypothesis} and the
    \textbf{conclusions}\footnote{You can include some footnotes.}.

    Include an icon that will help us remember the study (you can make it
    yourself or search on the internet). The idea is to make a visual connection
    with the research so we can say: \emph{``Remember the paper with the cute cat? We can
    use that analysis!''}.

    You can fill the document until about the bottom of this paragraph.
    Filler
    text text text text text text text text text text
    text text text text text text text text text text
    text text text text text text text text text text
    text text text text text text text text text text
    text text text text text text text text text text
    text text text text text text text text text text
    text text text text text text text text text text
    text text text text text text text text text text
    text text text text text text text text text text
    text text text text text text text text text text
    text.
\end{novelty}


\begin{tools}
    Did the author(s) use cool surveys or statistical analyses? We want to know
    too!

    \begin{itemize}
        \item You can
        \item Copy paste
        \item This item list
    \end{itemize}
\end{tools}

\begin{extra-info}
    Did you find an interesting theory? Was the experiment really interesting
    and you want to make it yourself? Tell us about it!

    Try to keep all the information you want to share within this page.
\end{extra-info}

\printbibliography[title=Summary From:]

\end{document}
